\documentclass[]{scrartcl}


%packages
\usepackage{amsmath}
\usepackage{xcolor}
\usepackage{hyperref}
\hypersetup{
	colorlinks   = true, %Colours links instead of ugly boxes
	urlcolor = teal
}

%opening
\title{Notes on Kalman filter for PTA analysis}
\author{}

\begin{document}

\maketitle

\begin{abstract}
This note defines the Kalman ``machinery"" for the state-space / residuals formulation of PTA analysis. It is heavily based on work done by P.Meyers and the \textsc{minnow} package, with extension to handle multiple pulsars and the influence of the stochastic GW background. 
\end{abstract}

\section{Requirements}

For the purposes of running state-space algorithms, we need to define the following:

\begin{itemize}
\item $\boldsymbol{X}$ : the state vector, dimension $n_X$
\item  $\boldsymbol{Y}$ : the observation vector, dimension $n_Y$
\item $\boldsymbol{F}$ : the state transition matrix, dimension $n_X$ $\times$ $n_X$
\item  $\boldsymbol{H}$ : the measurement matrix, dimension $n_Y$ $\times$ $n_X$
\item $\boldsymbol{Q}$ : the process noise covariance matrix, dimension $n_X$ $\times$ $n_X$
\item $\boldsymbol{R}$ : the measurement noise covariance matrix, dimension $n_Y$ $\times$ $n_Y$
\end{itemize}








\section{Single pulsar}\label{sec:single_pulsar}
To start, lets work with a single pulsar, $N=1$. \newline 


\subsection{State vector}

\noindent The state vector is 

\begin{equation}
	\boldsymbol{X} = \left(\delta \phi, \delta f, r,a,\delta \epsilon_1, \delta \epsilon_2, \cdots, \delta \epsilon_{\mathrm M} \right)
\end{equation}
The first two variables, $\delta \phi$ and $\delta f$, are deviations from the spin-down parameters in timing model fit. Since we know these deviations won’t move us more than 1 turn away (because these are nice millisecond pulsars) these are reasonable state variables to use. So for example, $\delta f$ is a fluctuation from the measured spin-down in the timing model. \newline 


\noindent The subsequent two variables, $r$ and $a$, quantify the effect of the gravitational wave. The variable $a(t)$ is the familiar redshift quantity from \href{https://arxiv.org/abs/2501.06990}{PTA P3}, and $r$ is the induced timing residual, obtained by integrating the redshift (see e.g. \href{https://arxiv.org/abs/1003.0677}{Equation 5 of Sesana \& vecchio 2010}). \newline 


\noindent The final $M$ parameters, $\delta \epsilon_i$ are parameters of the design matrix. These parameters will be used to incorporate the effects of the timing model. \newline 



\noindent The observation vector is 
\begin{equation}
	\boldsymbol{Y} = \left(\delta t \right)
\end{equation}
where $\delta t$ is the timing residual.


\subsection{Dynamical Equations (continuous time)}

The above state variables evolve according to the following dynamical equations 


\begin{align}
	\frac{d}{dt} \delta \phi^{(n)} &= \delta f^{(n)}, \\
	\frac{d}{dt} \delta f^{(n)} &= -\gamma_p^{(n)} \delta f^{(n)} + \chi_p^{(n)}(t;\sigma_p^{(n)}), \\
	\frac{d}{dt} r^{(n)} &= a^{(n)}, \\
	\frac{d}{dt} a^{(n)} &= -\gamma_a a^{(n)} + \chi_a^{(n)}(t;\sigma_a^{(n)}), \\
	\frac{d}{dt} \delta \epsilon^{(n)}_{m} &= \chi_\epsilon^{(n)}(t;\sigma_\epsilon) \quad \forall m \in [1, M^{(n)}].
\end{align}
where $\chi_a$ is the usual white noise stochastic process. \newline 



\noindent The states are related to the observables via a measurement equation
\begin{equation}
	\delta t = \frac{\delta \phi}{f_0} + \mathbf{M} \mathbf{\delta \epsilon} - r
\end{equation}
where  $\mathbf{M}$ is the \textit{design matrix} and $f_0$ is the pulsar rotation frequency \footnote{I need to check the sign of the residual term. Is it definitely a minus?}. \newline 


\noindent Note that the dynamics of the $\delta \epsilon^{(n)}_{m}$ terms here are a bit different to how it is done in \textsc{minnow}. In \textsc{minnow} these terms have zero process noise, and they just get initialised with some values (and covariance) which can then be estimated by the inference procedure. In our case I think we can marginalise over these timing model parameters by including them in the state and setting the variance $\sigma_\epsilon$ to be very large. This is effecively what pulsar people already do when using Gaussian processes in \textsc{enterprise}; see e.g. \href{https://arxiv.org/abs/1407.1838}{Section IV of van Haasteran \& Vallisneri, 2014}. In this way, we can avoid having to estimate an extra $\sum_{i=1}^{N} M^{(i)}$ parameters. I include the process noise term here by default; if we want to remove it in the future and do the full parameter estimation, we can just set $\sigma_\epsilon=0$ everywhere that follows.




\subsection{Discretisation}

\subsubsection{F-matrix}
	
	For the $\left(\delta\phi,\delta f\right)$ block, the continuous system is
	\begin{equation}
	\frac{d}{dt}\begin{pmatrix}\delta\phi \\ \delta f\end{pmatrix} 
	=
	\begin{pmatrix}
		0 & 1\\[1mm]
		0 & -\gamma_p
	\end{pmatrix}
	\begin{pmatrix}\delta\phi \\ \delta f\end{pmatrix}
	+
	\begin{pmatrix}0\\1\end{pmatrix}\chi_p(t),
	\end{equation}
	and the exact discretisation (matrix exponential) gives
	\begin{equation}
	F_p = \begin{pmatrix}
		1 & \dfrac{1-e^{-\gamma_p\Delta t}}{\gamma_p}\\[1mm]
		0 & e^{-\gamma_p\Delta t}
	\end{pmatrix}.
	\end{equation}
	Similarly, for the $\left(r,a\right)$ block we have
	\begin{equation}
	F_a = \begin{pmatrix}
		1 & \dfrac{1-e^{-\gamma_a\Delta t}}{\gamma_a}\\[1mm]
		0 & e^{-\gamma_a\Delta t}
	\end{pmatrix}.
	\end{equation}
For the timing model parameters $\delta\epsilon_m$, the dynamics are a random walk:
	\begin{equation}
	\delta\epsilon_{m,k+1} = \delta\epsilon_{m,k} + \eta_{\epsilon,m,k},
	\end{equation}
	so that the deterministic evolution is simply the identity. \newline 
	
	\noindent Thus, the overall discrete state transition matrix is given by the block diagonal matrix
	\begin{equation}
	\boldsymbol{F} = \begin{pmatrix}
		F_p & 0 & 0\\[1mm]
		0 & F_a & 0\\[1mm]
		0 & 0 & I_M
	\end{pmatrix},
	\end{equation}

	
\subsubsection{Q-matrix}
	
	For the $\left(\delta\phi,\delta f\right)$ block with noise variance $\sigma_p^2$, the discretised noise covariance is
	\begin{equation}
	Q_p = \sigma_p^2
	\begin{pmatrix}
		\displaystyle \frac{\Delta t}{\gamma_p^2} - \frac{2\left(1-e^{-\gamma_p\Delta t}\right)}{\gamma_p^3} + \frac{1-e^{-2\gamma_p\Delta t}}{2\gamma_p^3} & \displaystyle \frac{1-e^{-\gamma_p\Delta t}}{\gamma_p^2} - \frac{1-e^{-2\gamma_p\Delta t}}{2\gamma_p^2}\\[2mm]
		\displaystyle \frac{1-e^{-\gamma_p\Delta t}}{\gamma_p^2} - \frac{1-e^{-2\gamma_p\Delta t}}{2\gamma_p^2} & \displaystyle \frac{1-e^{-2\gamma_p\Delta t}}{2\gamma_p}
	\end{pmatrix}.
	\end{equation}
	For the $\left(r,a\right)$ block with noise variance $\sigma_a^2$, we similarly have
	\begin{equation}
	Q_a = \sigma_a^2
	\begin{pmatrix}
		\displaystyle \frac{\Delta t}{\gamma_a^2} - \frac{2\left(1-e^{-\gamma_a\Delta t}\right)}{\gamma_a^3} + \frac{1-e^{-2\gamma_a\Delta t}}{2\gamma_a^3} & \displaystyle \frac{1-e^{-\gamma_a\Delta t}}{\gamma_a^2} - \frac{1-e^{-2\gamma_a\Delta t}}{2\gamma_a^2}\\[2mm]
		\displaystyle \frac{1-e^{-\gamma_a\Delta t}}{\gamma_a^2} - \frac{1-e^{-2\gamma_a\Delta t}}{2\gamma_a^2} & \displaystyle \frac{1-e^{-2\gamma_a\Delta t}}{2\gamma_a}
	\end{pmatrix}.
	\end{equation}
	For each timing model parameter $\delta\epsilon_m$, modeled as a random walk,
	\begin{equation}
	\delta\epsilon_{m,k+1} = \delta\epsilon_{m,k} + \eta_{\epsilon,m,k}, \quad \eta_{\epsilon,m,k}\sim \mathcal{N}(0,\sigma_\epsilon^2\,\Delta t),
	\end{equation}
	so that
	\begin{equation}
	Q_\epsilon = \sigma_\epsilon^2\,\Delta t\,I_M.
	\end{equation}
	Thus, the overall process noise covariance is block diagonal:
	\begin{equation}
	\boldsymbol{Q} = \begin{pmatrix}
		Q_p & 0 & 0\\[2mm]
		0 & Q_a & 0\\[2mm]
		0 & 0 & \sigma_\epsilon^2\,\Delta t\,I_M
	\end{pmatrix}.
	\end{equation}
	
\subsubsection{H-matrix}
	Recall that the measurement equation is
		\begin{equation}
	\delta t = \frac{\delta\phi}{f_0} + \boldsymbol{M}\,\boldsymbol{\delta\epsilon} - r.
		\end{equation}
	Since the state vector is
	\begin{equation}
	\boldsymbol{X} = \begin{pmatrix}
		\delta\phi \\ \delta f \\ r \\ a \\ \delta\epsilon_1 \\ \vdots \\ \delta\epsilon_M
	\end{pmatrix},
		\end{equation}
	the measurement depends only on $\delta\phi$, $r$, and the $\delta\epsilon_m$. Therefore, the measurement matrix is
		\begin{equation}
	\boldsymbol{H} = \begin{pmatrix}
		\frac{1}{f_0} & 0 & -1 & 0 & M_1 & \cdots & M_M
	\end{pmatrix},
		\end{equation}
	so that
		\begin{equation}
	\delta t = \boldsymbol{H}\,\boldsymbol{X}.
		\end{equation}
	
\subsubsection{R-matrix}
	Assuming the measurement noise is white with variance $\sigma_t^2$, and given that there is a single measurement per time step, we have
	\begin{equation}
	\boldsymbol{R} = \sigma_t^2.
	\end{equation}

	


\section{Multiple pulsars}

The formulation in Section \ref{sec:single_pulsar} extends straightforwardly to multiple pulsars. There are two complications which must be handled with care.

\begin{enumerate}
	\item The covariance between the $a^{(n)}$ terms for different pulsars. This is the Hellings-Downs effect.
	\item The fact that in general all pulsars are observed at different times. This means that instead of having a nice observation vector $\mathbf{Y}$ of length $N$, in general we just have a single observation from a single pulsar at a given timestep. 
\end{enumerate}


\noindent Regarding (1), the ensemble statistics of $\chi_{\mathrm a}^{(n)}$ are
\begin{align}
	\langle \chi^{(n)}_{\mathrm a}(t) \rangle &= 0 \ , 	\label{eq:xieqn1} \\
	\langle \chi^{(n)}_{\mathrm a}(t) \, \chi^{(n')}_{\mathrm a}(t') \rangle &= \left[\sigma^{(n,n')}_{\mathrm a}\right]^2 \delta(t - t') \ .	\label{eq:xieqn2}
\end{align}
with
\begin{eqnarray}
	\left[\sigma^{(n,n')}_{\mathrm a}\right]^2 = \frac{\langle h^2\rangle}{6} \gamma_{\mathrm a} \, \Gamma \left[ \theta^{(n,n')} \right] \, , \label{eq:sigma_a_expression}
\end{eqnarray}
where $\langle h^2 \rangle$ is the mean square GW strain from the $M$ summed sources at the Earth's position, $\theta^{(n,n')}$ is the angle between the $n$-th and $n'$-th pulsars, and one has
\begin{eqnarray}
	\Gamma\left[\theta^{(n,n')} \right] =  \frac{3}{2} x_{n n'} \ln x_{n n'}  -\frac{x_{n n'} }{4}+\frac{1}{2} + \frac{1}{2} \delta_{n n'}\label{eq:correlation} \, ,
\end{eqnarray}
with $x_{nn'} = \left[1 - \cos \theta^{(n,n')}\right]/2$. 



\subsection{Discretisation for multiple pulsars}
	
	For $N$ pulsars, we define the stacked state vector
	\begin{equation}
		\boldsymbol{X} = \begin{pmatrix}
			\boldsymbol{X}^{(1)} \\
			\boldsymbol{X}^{(2)} \\
			\vdots \\
			\boldsymbol{X}^{(N)}
		\end{pmatrix}, \qquad \text{with} \qquad
		\boldsymbol{X}^{(n)} = \begin{pmatrix}
			\delta \phi^{(n)}\\[1mm]
			\delta f^{(n)}\\[1mm]
			r^{(n)}\\[1mm]
			a^{(n)}\\[1mm]
			\delta\epsilon_1^{(n)}\\[1mm]
			\vdots\\[1mm]
			\delta\epsilon_{M^{(n)}}^{(n)}
		\end{pmatrix}.
	\end{equation}
	The state evolution is as in the single--pulsar case, except that the gravitational-wave noise processes $\chi_a^{(n)}(t)$ are correlated among pulsars. Their ensemble statistics are
	\begin{equation}
		\langle \chi_a^{(n)}(t) \rangle = 0,
	\end{equation}
	\begin{equation}
		\langle \chi_a^{(n)}(t)\,\chi_a^{(n')}(t') \rangle = \left[\sigma_a^{(n,n')}\right]^2\,\delta(t-t'),
	\end{equation}
	with
	\begin{equation}
		\left[\sigma_a^{(n,n')}\right]^2 = \frac{\langle h^2 \rangle}{6}\,\gamma_a\,\Gamma\bigl[\theta^{(n,n')}\bigr],
	\end{equation}
	and
	\begin{equation}
		\Gamma\bigl[\theta^{(n,n')}\bigr] = \frac{3}{2}x_{nn'}\ln x_{nn'} - \frac{x_{nn'}}{4} + \frac{1}{2} + \frac{1}{2}\delta_{nn'}, \qquad x_{nn'}=\frac{1-\cos\theta^{(n,n')}}{2}\,.
	\end{equation}
	
	\subsection*{1. Discretised State Transition Matrix (\(\boldsymbol{F}\))}
	
	For pulsar $n$, the \((\delta\phi,\delta f)\) block discretises as
	\begin{equation}
		F_p^{(n)} = \begin{pmatrix}
			1 & \dfrac{1-e^{-\gamma_p^{(n)}\Delta t}}{\gamma_p^{(n)}} \\[1mm]
			0 & e^{-\gamma_p^{(n)}\Delta t}
		\end{pmatrix},
	\end{equation}
	and the \((r,a)\) block is
	\begin{equation}
		F_a = \begin{pmatrix}
			1 & \dfrac{1-e^{-\gamma_a\Delta t}}{\gamma_a} \\[1mm]
			0 & e^{-\gamma_a\Delta t}
		\end{pmatrix},
	\end{equation}
	(with the same gravitational–wave damping rate $\gamma_a$ assumed for all pulsars). The timing model parameters evolve by the identity. Hence, the state transition matrix for pulsar $n$ is
	\begin{equation}
		F^{(n)} = \begin{pmatrix}
			F_p^{(n)} & 0 & 0\\[2mm]
			0 & F_a & 0\\[2mm]
			0 & 0 & I_{M^{(n)}}
		\end{pmatrix}.
	\end{equation}
	Stacking the pulsar states, the overall state transition matrix is block diagonal:
	\begin{equation}
		\boldsymbol{F} = \mathrm{diag}\Bigl\{F^{(1)},F^{(2)},\ldots,F^{(N)}\Bigr\}\,.
	\end{equation}
	
	\subsection*{2. Discretised Process Noise Covariance (\(\boldsymbol{Q}\))}
	
	For pulsar $n$, the \((\delta\phi,\delta f)\) block has the discretised covariance
	\begin{equation}
		Q_p^{(n)} = \sigma_p^{(n)2}\begin{pmatrix}
			\displaystyle \frac{\Delta t}{\gamma_p^{(n)2}} - \frac{2\left(1-e^{-\gamma_p^{(n)}\Delta t}\right)}{\gamma_p^{(n)3}} + \frac{1-e^{-2\gamma_p^{(n)}\Delta t}}{2\gamma_p^{(n)3}} & \displaystyle \frac{1-e^{-\gamma_p^{(n)}\Delta t}}{\gamma_p^{(n)2}} - \frac{1-e^{-2\gamma_p^{(n)}\Delta t}}{2\gamma_p^{(n)2}}\\[2mm]
			\displaystyle \frac{1-e^{-\gamma_p^{(n)}\Delta t}}{\gamma_p^{(n)2}} - \frac{1-e^{-2\gamma_p^{(n)}\Delta t}}{2\gamma_p^{(n)2}} & \displaystyle \frac{1-e^{-2\gamma_p^{(n)}\Delta t}}{2\gamma_p^{(n)}}
		\end{pmatrix}.
	\end{equation}
	For the gravitational--wave (or \(a\)) block, note that the continuous noise processes are correlated among different pulsars. Thus, the cross–covariance between pulsars $n$ and $n'$ is discretised as
	\begin{equation}
		Q_a^{(n,n')} = \left[\sigma_a^{(n,n')}\right]^2\begin{pmatrix}
			\displaystyle \frac{\Delta t}{\gamma_a^2} - \frac{2\left(1-e^{-\gamma_a\Delta t}\right)}{\gamma_a^3} + \frac{1-e^{-2\gamma_a\Delta t}}{2\gamma_a^3} & \displaystyle \frac{1-e^{-\gamma_a\Delta t}}{\gamma_a^2} - \frac{1-e^{-2\gamma_a\Delta t}}{2\gamma_a^2}\\[2mm]
			\displaystyle \frac{1-e^{-\gamma_a\Delta t}}{\gamma_a^2} - \frac{1-e^{-2\gamma_a\Delta t}}{2\gamma_a^2} & \displaystyle \frac{1-e^{-2\gamma_a\Delta t}}{2\gamma_a}
		\end{pmatrix}.
	\end{equation}
	For the timing model parameters of pulsar $n$, modeled as a random walk,
	\begin{equation}
		Q_\epsilon^{(n)} = \sigma_\epsilon^2\,\Delta t\,I_{M^{(n)}}.
	\end{equation}
	
	The block coupling pulsars $n$ and $n'$ is then assembled as
	\begin{equation}
		\boldsymbol{Q}^{(n,n')} = \begin{pmatrix}
			\delta_{nn'}\, Q_p^{(n)} & 0 & 0 \\[2mm]
			0 & Q_a^{(n,n')} & 0 \\[2mm]
			0 & 0 & \delta_{nn'}\, Q_\epsilon^{(n)}
		\end{pmatrix},
	\end{equation}
	where $\delta_{nn'}$ is the Kronecker delta (i.e. the spin and timing noise are uncorrelated between different pulsars). Finally, the full process noise covariance is the block matrix
	\begin{equation}
		\boldsymbol{Q} = \begin{pmatrix}
			\boldsymbol{Q}^{(1,1)} & \boldsymbol{Q}^{(1,2)} & \cdots & \boldsymbol{Q}^{(1,N)} \\[1mm]
			\boldsymbol{Q}^{(2,1)} & \boldsymbol{Q}^{(2,2)} & \cdots & \boldsymbol{Q}^{(2,N)} \\[1mm]
			\vdots & \vdots & \ddots & \vdots \\[1mm]
			\boldsymbol{Q}^{(N,1)} & \boldsymbol{Q}^{(N,2)} & \cdots & \boldsymbol{Q}^{(N,N)}
		\end{pmatrix}\,.
	\end{equation}
	
	\subsection*{3. Measurement Matrix (\(\boldsymbol{H}\))}
	
	For pulsar $n$, the measurement equation is
	\begin{equation}
		\delta t^{(n)} = \frac{\delta\phi^{(n)}}{f_0^{(n)}} + \boldsymbol{M}^{(n)}\,\boldsymbol{\delta\epsilon}^{(n)} - r^{(n)}\,,
	\end{equation}
	so that the measurement matrix is
	\begin{equation}
		\boldsymbol{H}^{(n)} = \begin{pmatrix}
			\frac{1}{f_0^{(n)}} & 0 & -1 & 0 & M_1^{(n)} & \cdots & M_{M^{(n)}}^{(n)}
		\end{pmatrix}\,.
	\end{equation}
	In a multi–pulsar analysis the overall measurement vector $\boldsymbol{Y}$ is built by stacking the individual measurements. In practice, since pulsars are generally observed at different times, only a subset of the pulsars contribute at any given time; the full measurement matrix is then formed by selecting the corresponding rows from the block–diagonal matrix
	\begin{equation}
		\boldsymbol{H}_{\mathrm{full}} = \mathrm{diag}\Bigl\{\boldsymbol{H}^{(1)},\boldsymbol{H}^{(2)},\ldots,\boldsymbol{H}^{(N)}\Bigr\}\,.
	\end{equation}
	
	\subsection*{4. Measurement Noise Covariance (\(\boldsymbol{R}\))}
	
	Assuming that the measurement noise for pulsar $n$ is white with variance $\sigma_t^{(n)2}$ and that different pulsars have independent measurement noise, then if at a given time step the set of observed pulsars is $\mathcal{O}\subset\{1,\ldots,N\}$, the measurement noise covariance is
	\begin{equation}
		\boldsymbol{R} = \mathrm{diag}\Bigl\{\sigma_t^{(n)2}\Bigr\}_{n\in\mathcal{O}}\,.
	\end{equation}


	
%\section{additional notes}
%
%It may help the coding if we reorder some things...
%
%\section{Pipeline}
%
%\begin{enumerate}
%	\item Obtain a .par and .tim file for a pulsar
%	\item Pass these files through TEMPO/PINT to obtain timing resiudals $\delta t$, and a design matrix $\mathbf{M}$.
%\end{enumerate}
%
%\section{Dispersion and multiband}
%
%all of the above assumes wideband


\newpage
\appendix
\section{A worked example for the Q-matrix}


It can help intution to "see" what the Q-matrix looks like explicitly. Consider the case where $N=2$. \newline 



	Recall that for each pulsar the state vector is partitioned into three sectors:
	\begin{enumerate}
		\item \textbf{Spin noise sector} (for \(\delta\phi\) and \(\delta f\)): a \(2\times2\) block.
		\item \textbf{Gravitational--wave (redshift/residual) sector} (for \(r\) and \(a\)): a \(2\times2\) block.
		\item \textbf{Timing model (design matrix) sector} (for \(\delta\epsilon\)): a block of size \(M^{(n)}\times M^{(n)}\).
	\end{enumerate}
	
	For pulsar \(n\) (\(n=1,2\)), the individual blocks are defined as follows:
	
	\subsection*{Spin Noise Sector}
	
	For pulsar \(n\), the discretised spin noise covariance is
	\begin{equation}
		Q_p^{(n)} = \sigma_p^{(n)2}
		\begin{pmatrix}
			\displaystyle \frac{\Delta t}{\gamma_p^{(n)2}} - \frac{2\bigl(1-e^{-\gamma_p^{(n)}\Delta t}\bigr)}{\gamma_p^{(n)3}} + \frac{1-e^{-2\gamma_p^{(n)}\Delta t}}{2\gamma_p^{(n)3}} 
			& \displaystyle \frac{1-e^{-\gamma_p^{(n)}\Delta t}}{\gamma_p^{(n)2}} - \frac{1-e^{-2\gamma_p^{(n)}\Delta t}}{2\gamma_p^{(n)2}} \\[2mm]
			\displaystyle \frac{1-e^{-\gamma_p^{(n)}\Delta t}}{\gamma_p^{(n)2}} - \frac{1-e^{-2\gamma_p^{(n)}\Delta t}}{2\gamma_p^{(n)2}} 
			& \displaystyle \frac{1-e^{-2\gamma_p^{(n)}\Delta t}}{2\gamma_p^{(n)}}
		\end{pmatrix}\,.
	\end{equation}
	
	\subsection*{Gravitational--Wave Sector}
	
	For the gravitational--wave noise, the continuous noise processes \(\chi_a^{(n)}(t)\) are correlated among pulsars. Thus, for pulsars \(n\) and \(n'\) the discretised covariance is given by:
	\begin{equation}
		Q_a^{(n,n')} = \left[\sigma_a^{(n,n')}\right]^2
		\begin{pmatrix}
			\displaystyle \frac{\Delta t}{\gamma_a^2} - \frac{2\bigl(1-e^{-\gamma_a\Delta t}\bigr)}{\gamma_a^3} + \frac{1-e^{-2\gamma_a\Delta t}}{2\gamma_a^3} 
			& \displaystyle \frac{1-e^{-\gamma_a\Delta t}}{\gamma_a^2} - \frac{1-e^{-2\gamma_a\Delta t}}{2\gamma_a^2} \\[2mm]
			\displaystyle \frac{1-e^{-\gamma_a\Delta t}}{\gamma_a^2} - \frac{1-e^{-2\gamma_a\Delta t}}{2\gamma_a^2} 
			& \displaystyle \frac{1-e^{-2\gamma_a\Delta t}}{2\gamma_a}
		\end{pmatrix}\,.
	\end{equation}
	
	For the \textbf{autocovariance} (i.e. \(n=n'\)) we denote this as \(Q_a^{(n,n)}\), and for the \textbf{cross-covariance} (i.e. \(n \neq n'\)) we have
	\[
	Q_a^{(1,2)} = Q_a^{(2,1)}.
	\]
	
	\subsection*{Timing Model Sector}
	
	For pulsar \(n\), the timing model parameters are assumed to follow a random walk. Their covariance is given by
	\begin{equation}
		Q_\epsilon^{(n)} = \sigma_\epsilon^2\,\Delta t\,I_{M^{(n)}},
	\end{equation}
	where \(I_{M^{(n)}}\) is the identity matrix of dimension \(M^{(n)}\).
	
	\subsection*{Constructing the Full \(\boldsymbol{Q}\) Matrix for Two Pulsars}
	
	For two pulsars (\(N=2\)), we construct the overall process noise covariance matrix \(\boldsymbol{Q}\) as a \(2\times2\) block matrix:
	\begin{equation}
		\boldsymbol{Q} =
		\begin{pmatrix}
			\boldsymbol{Q}^{(1,1)} & \boldsymbol{Q}^{(1,2)} \\[1mm]
			\boldsymbol{Q}^{(2,1)} & \boldsymbol{Q}^{(2,2)}
		\end{pmatrix}\,.
	\end{equation}
	
	Each block \(\boldsymbol{Q}^{(n,n')}\) is itself block–structured into three sectors (spin noise, gravitational–wave noise, and timing model noise).
	
	\subsubsection*{Diagonal Block for Pulsar 1 (\(\boldsymbol{Q}^{(1,1)}\))}
	
	\begin{equation}
		\boldsymbol{Q}^{(1,1)} =
		\begin{pmatrix}
			Q_p^{(1)} & \boldsymbol{0}_{2\times2} & \boldsymbol{0}_{2\times M^{(1)}} \\[2mm]
			\boldsymbol{0}_{2\times2} & Q_a^{(1,1)} & \boldsymbol{0}_{2\times M^{(1)}} \\[2mm]
			\boldsymbol{0}_{M^{(1)}\times2} & \boldsymbol{0}_{M^{(1)}\times2} & Q_\epsilon^{(1)}
		\end{pmatrix}\,.
	\end{equation}
	
	Here:
	\begin{itemize}
		\item The upper-left \(2\times2\) block is \(Q_p^{(1)}\) (spin noise for pulsar 1).
		\item The middle \(2\times2\) block is \(Q_a^{(1,1)}\) (gravitational–wave autocovariance for pulsar 1).
		\item The lower-right \(M^{(1)}\times M^{(1)}\) block is \(Q_\epsilon^{(1)}\) (timing model noise for pulsar 1).
	\end{itemize}
	
	\subsubsection*{Diagonal Block for Pulsar 2 (\(\boldsymbol{Q}^{(2,2)}\))}
	
	\begin{equation}
		\boldsymbol{Q}^{(2,2)} =
		\begin{pmatrix}
			Q_p^{(2)} & \boldsymbol{0}_{2\times2} & \boldsymbol{0}_{2\times M^{(2)}} \\[2mm]
			\boldsymbol{0}_{2\times2} & Q_a^{(2,2)} & \boldsymbol{0}_{2\times M^{(2)}} \\[2mm]
			\boldsymbol{0}_{M^{(2)}\times2} & \boldsymbol{0}_{M^{(2)}\times2} & Q_\epsilon^{(2)}
		\end{pmatrix}\,.
	\end{equation}
	
	Here:
	\begin{itemize}
		\item The upper-left \(2\times2\) block is \(Q_p^{(2)}\) (spin noise for pulsar 2).
		\item The middle \(2\times2\) block is \(Q_a^{(2,2)}\) (gravitational–wave autocovariance for pulsar 2).
		\item The lower-right \(M^{(2)}\times M^{(2)}\) block is \(Q_\epsilon^{(2)}\) (timing model noise for pulsar 2).
	\end{itemize}
	
	\subsubsection*{Off-Diagonal Block Between Pulsar 1 and Pulsar 2 (\(\boldsymbol{Q}^{(1,2)}\))}
	
	\begin{equation}
		\boldsymbol{Q}^{(1,2)} =
		\begin{pmatrix}
			\boldsymbol{0}_{2\times2} & \boldsymbol{0}_{2\times2} & \boldsymbol{0}_{2\times M^{(2)}} \\[2mm]
			\boldsymbol{0}_{2\times2} & Q_a^{(1,2)} & \boldsymbol{0}_{2\times M^{(2)}} \\[2mm]
			\boldsymbol{0}_{M^{(1)}\times2} & \boldsymbol{0}_{M^{(1)}\times2} & \boldsymbol{0}_{M^{(1)}\times M^{(2)}}
		\end{pmatrix}\,.
	\end{equation}
	
	In this off-diagonal block:
	\begin{itemize}
		\item The spin noise sectors (upper-left \(2\times2\)) are zero because spin noise is uncorrelated between pulsars.
		\item The timing model sectors are zero.
		\item Only the gravitational–wave sector (the middle \(2\times2\) block) is nonzero and given by \(Q_a^{(1,2)}\).
	\end{itemize}
	
	\subsubsection*{Off-Diagonal Block Between Pulsar 2 and Pulsar 1 (\(\boldsymbol{Q}^{(2,1)}\))}
	
	By symmetry (assuming the cross–covariance is symmetric), we have
	\begin{equation}
		\boldsymbol{Q}^{(2,1)} =
		\begin{pmatrix}
			\boldsymbol{0}_{2\times2} & \boldsymbol{0}_{2\times2} & \boldsymbol{0}_{2\times M^{(1)}} \\[2mm]
			\boldsymbol{0}_{2\times2} & Q_a^{(2,1)} & \boldsymbol{0}_{2\times M^{(1)}} \\[2mm]
			\boldsymbol{0}_{M^{(2)}\times2} & \boldsymbol{0}_{M^{(2)}\times2} & \boldsymbol{0}_{M^{(2)}\times M^{(1)}}
		\end{pmatrix}\,,
	\end{equation}
	with \(Q_a^{(2,1)} = Q_a^{(1,2)}\).
	
	\subsection*{Full \(\boldsymbol{Q}\) Matrix for \(N=2\)}
	
	Assembling the blocks, the full \( \boldsymbol{Q} \) matrix is given by
	\begin{equation}
		\boldsymbol{Q} =
		\begin{pmatrix}
			\begin{array}{c|c}
				\begin{array}{ccc}
					Q_p^{(1)} & \boldsymbol{0}_{2\times2} & \boldsymbol{0}_{2\times M^{(1)}}
					\\[2mm]
					\boldsymbol{0}_{2\times2} & Q_a^{(1,1)} & \boldsymbol{0}_{2\times M^{(1)}}
					\\[2mm]
					\boldsymbol{0}_{M^{(1)}\times2} & \boldsymbol{0}_{M^{(1)}\times2} & Q_\epsilon^{(1)}
				\end{array}
				& 
				\begin{array}{ccc}
					\boldsymbol{0}_{2\times2} & \boldsymbol{0}_{2\times2} & \boldsymbol{0}_{2\times M^{(2)}
					}\\[2mm]
					\boldsymbol{0}_{2\times2} & Q_a^{(1,2)} & \boldsymbol{0}_{2\times M^{(2)}
					}\\[2mm]
					\boldsymbol{0}_{M^{(1)}\times2} & \boldsymbol{0}_{M^{(1)}\times2} & \boldsymbol{0}_{M^{(1)}\times M^{(2)}
					}
				\end{array}
				\\[4mm]
				\hline \\[-3mm]
				\begin{array}{ccc}
					\boldsymbol{0}_{2\times2} & \boldsymbol{0}_{2\times2} & \boldsymbol{0}_{2\times M^{(1)}
					}\\[2mm]
					\boldsymbol{0}_{2\times2} & Q_a^{(2,1)} & \boldsymbol{0}_{2\times M^{(1)}
					}\\[2mm]
					\boldsymbol{0}_{M^{(2)}\times2} & \boldsymbol{0}_{M^{(2)}\times2} & \boldsymbol{0}_{M^{(2)}\times M^{(1)}
					}
				\end{array}
				&
				\begin{array}{ccc}
					Q_p^{(2)} & \boldsymbol{0}_{2\times2} & \boldsymbol{0}_{2\times M^{(2)}
					}\\[2mm]
					\boldsymbol{0}_{2\times2} & Q_a^{(2,2)} & \boldsymbol{0}_{2\times M^{(2)}
					}\\[2mm]
					\boldsymbol{0}_{M^{(2)}\times2} & \boldsymbol{0}_{M^{(2)}\times2} & Q_\epsilon^{(2)}
				\end{array}
			\end{array}
		\end{pmatrix}\,.
	\end{equation}
	
	\textbf{Component Summary:}
	\begin{itemize}
		\item \(\boldsymbol{Q}^{(1,1)}\) (Pulsar 1):
		\begin{itemize}
			\item \textbf{Spin noise:} \(Q_p^{(1)}\) (upper-left \(2\times2\))
			\item \textbf{Gravitational--wave noise:} \(Q_a^{(1,1)}\) (middle \(2\times2\))
			\item \textbf{Timing model noise:} \(Q_\epsilon^{(1)}\) (lower-right \(M^{(1)}\times M^{(1)}\))
		\end{itemize}
		\item \(\boldsymbol{Q}^{(2,2)}\) (Pulsar 2):
		\begin{itemize}
			\item \textbf{Spin noise:} \(Q_p^{(2)}\)
			\item \textbf{Gravitational--wave noise:} \(Q_a^{(2,2)}\)
			\item \textbf{Timing model noise:} \(Q_\epsilon^{(2)}\)
		\end{itemize}
		\item \(\boldsymbol{Q}^{(1,2)}\) (Between Pulsar 1 and Pulsar 2):
		\begin{itemize}
			\item \textbf{Spin noise:} Zero (\(2\times2\) zero matrix)
			\item \textbf{Gravitational--wave noise:} \(Q_a^{(1,2)}\) (middle \(2\times2\))
			\item \textbf{Timing model noise:} Zero
		\end{itemize}
		\item \(\boldsymbol{Q}^{(2,1)}\) (Between Pulsar 2 and Pulsar 1):
		\begin{itemize}
			\item \textbf{Spin noise:} Zero
			\item \textbf{Gravitational--wave noise:} \(Q_a^{(2,1)}\) (which equals \(Q_a^{(1,2)}\))
			\item \textbf{Timing model noise:} Zero
		\end{itemize}
	\end{itemize}
	
	This explicit layout shows how the overall \( \boldsymbol{Q} \) matrix incorporates:
	\begin{itemize}
		\item The individual (uncorrelated) spin noise and timing model noise (appearing in the diagonal blocks).
		\item The cross–correlations due to the gravitational--wave background (appearing in the off-diagonal blocks).
	\end{itemize}
	



\end{document}
